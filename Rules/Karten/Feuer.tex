\subsection{Feuer}

\subsubsection{Feuer mit unterschiedlichen Waffen}
Waffen des Types 'Fernkampf' können, wenn ausgerüstet, die Regeln für Feuer abändern. Die ensprechenden
Änderungen werden auf der jeweiligen Waffenkarte aufgelistet.

\subsubsection{Munitionsverbrauch}
Um die Aktion 'Feuer' ausführen zu können, muss der Spieler über mindestes so viel Munition verfügen wie
wie die ausgerüstete Waffer benötigt. Hast du nicht genügent Munition, kannst du nicht Schießen und deine Aktion ferfällt.
Den Munitionsverbrauch einer Waffe findest du auf der Waffenkarte.
Deine Munition wird durch die Munitionskarten dargestellt. Verbrauchst du Munition, lege so viele deiner Munitionskarten
auf den Verbraucht Munitionsstapel wie auf der verwendeten Waffenkarte beschriebn ist.

\subsubsection{Waffe fallenlassen}
Du kannst dich beim ausführen der Feuer-Aktion entscheiden eine ausgerüstete Waffe fallen zu lassen. Wenn du das tust,
plaziere die Waffenmarke auf dem Feld, auf dem sich dein Bandiet befindet. Du darft anschließend die Aktion mit der
Standartwaffe ausführen. Die Standartwaffe kann nicht fallen gelassen werden. Ein Grund für das fallenlassen einer
Waffe kann sein, das die ausgerüstete Waffe mehr Munition verbraucht als gerade vorhanden, wärend die Munition für
das Abfeuern der Standartwaffe ausreicht.

\subsubsection{Wunden zuteilung}
Der Spieler wählt den Banditen eines Mitspielers als
Ziel aus und gibt ihm eine Wundenkarten. Dieser legt die Wundenkarte auf sein persönliches
Deck.