\subsection{Nahkampfwaffen}


\begin{table}[H]
  \begin{center}
    \label{tab:table1}
    \begin{tabular}{|l|p{8cm}|}
      \hline
      \textbf{Name} & Fäuste \\
      \hline
      \textbf{Art} & Nahkampf, Standart \\
      \hline
      \textbf{Munitionsverbrauch} & 0 \\
      \hline
      \textbf{Effekt} & - \\
      \hline
      \textbf{Fluftext} & - \\
      \hline
    \end{tabular}
  \end{center}
\end{table}


\begin{table}[H]
  \begin{center}
    \label{tab:table1}
    \begin{tabular}{|l|p{8cm}|}
      \hline
      \textbf{Name} & Schlagring \\
      \hline
      \textbf{Art} & Nahkampf \\
      \hline
      \textbf{Munitionsverbrauch} & 0 \\
      \hline
      \textbf{Effekt} & Das Ziel kann zusätzlich nach Oben beziehungsweise Unten bewegt werden. Es ist nur
                        einen Bewegung erlaubt, ein Ziel kann nicht durch dies Waffe in einen anderen Wagon
                        und eine anderen Ebene bewegt werden \\
      \hline
      \textbf{Fluftext} & Uppercut des Todes \\
      \hline
    \end{tabular}
  \end{center}
\end{table}

\begin{table}[H]
  \begin{center}
    \label{tab:table1}
    \begin{tabular}{|l|p{8cm}|}
      \hline
      \textbf{Name} & Nunchaku \\
      \hline
      \textbf{Art} & Nahkampf \\
      \hline
      \textbf{Munitionsverbrauch} & 0 \\
      \hline
      \textbf{Effekt} & Am Ende deines Nahkampfangriffs, wenn weiter Ziel möglich sind: Wirf eine Münze,
                        Kopf: Du erhällst eine Wunde. Zahl: Du kannst einen weiteren Nahkampfangriff gegen
                        ein anderes erlaubtes Ziel ausführen. Wiederhole diesen Vorgang bis du keine Ziele
                        mehr hast oder eine Kopf geworfen hast \\
      \hline
      \textbf{Fluftext} & - \\
      \hline
    \end{tabular}
  \end{center}
\end{table}

\begin{table}[H]
  \begin{center}
    \label{tab:table1}
    \begin{tabular}{|l|p{8cm}|}
      \hline
      \textbf{Name} & Machete \\
      \hline
      \textbf{Art} & Nahkampf \\
      \hline
      \textbf{Munitionsverbrauch} & 0 \\
      \hline
      \textbf{Effekt} & Das Ziel erhällt zusätzlich ein Wunde \\
      \hline
      \textbf{Fluftext} & Machete schreibt keine SMS \\
      \hline
    \end{tabular}
  \end{center}
\end{table}

\begin{table}[H]
  \begin{center}
    \label{tab:table1}
    \begin{tabular}{|l|p{8cm}|}
      \hline
      \textbf{Name} & Lasso \\
      \hline
      \textbf{Art} & Nahkampf \\
      \hline
      \textbf{Munitionsverbrauch} & 0 \\
      \hline
      \textbf{Effekt} & Kann nur Spieler in Nachbarfeldern als Ziel nehmen. Ansatt das Ziel ein Feld in
                        beliebige Richtung zu bewegen, wähle:
                        A: Das Ziel wird ein Feld in deine Richtung bewegt, nachdem dieses die Beute
                           fallengelassen hat
                        B: Die fallengelassene Beute wird ein Feld in deine Richtung bewegt. Das Ziel
                           wird nicht bewegt \\
      \hline
      \textbf{Fluftext} & - \\
      \hline
    \end{tabular}
  \end{center}
\end{table}