


\subsection{Fernkampfwaffen}


\begin{table}[H]
  \begin{center}
    \label{tab:table1}
    \begin{tabular}{|l|p{8cm}|}
      \hline
      \textbf{Name} & Revolver \\
      \hline
      \textbf{Art} & Fernkampf, Standart \\
      \hline
      \textbf{Munitionsverbrauch} & 1 \\
      \hline
      \textbf{Effekt} & - \\
      \hline
      \textbf{Fluftext} & - \\
      \hline
    \end{tabular}
  \end{center}
\end{table}

\begin{table}[H]
  \begin{center}
    \label{tab:table1}
    \begin{tabular}{|l|p{8cm}|}
      \hline
      \textbf{Name} & Schrotflinte \\
      \hline
      \textbf{Art} & Fernkampf \\
      \hline
      \textbf{Munitionsverbrauch} & 2 \\
      \hline
      \textbf{Effekt} & Alle weiteren Spieler die sich auf dem selben Feld wie das Ziel befinden erhaletn
                        ebenfalls eine Wunde. Werden dadurch mindestens ein weiterer Spieler getroffen,
                        erhällt der Anwender eine Weitere Treffermarke \\
      \hline
      \textbf{Fluftext} & - \\
      \hline
    \end{tabular}
  \end{center}
\end{table}

\begin{table}[H]
  \begin{center}
    \label{tab:table1}
    \begin{tabular}{|l|p{8cm}|}
      \hline
      \textbf{Name} & Peitsche \\
      \hline
      \textbf{Art} & Fernkampf \\
      \hline
      \textbf{Munitionsverbrauch} & 0 \\
      \hline
      \textbf{Effekt} & Es können immer nur die Spieler als Ziel genommen werden, die sich auf einem Angrenzenden Feld
                        und der selben Ebene befinden\\
      \hline
      \textbf{Fluftext} & - \\
      \hline
    \end{tabular}
  \end{center}
\end{table}

\begin{table}[H]
  \begin{center}
    \label{tab:table1}
    \begin{tabular}{|l|p{8cm}|}
      \hline
      \textbf{Name} & --- \\
      \hline
      \textbf{Art} & Fernkampf \\
      \hline
      \textbf{Munitionsverbrauch} & 1 \\
      \hline
      \textbf{Effekt} & Spieler auf den Feldern über/unter dem Anwender können zusätzlich als Ziel gewählt
                        werden \\
      \hline
      \textbf{Fluftext} & - \\
      \hline
    \end{tabular}
  \end{center}
\end{table}

\begin{table}[H]
  \begin{center}
    \label{tab:table1}
    \begin{tabular}{|l|p{8cm}|}
      \hline
      \textbf{Name} & --- \\
      \hline
      \textbf{Art} & Fernkampf \\
      \hline
      \textbf{Munitionsverbrauch} & 1 \\
      \hline
      \textbf{Effekt} & Es können bis zu zwei erlaubte Ziele gewählt werden. Die Ziele müssen sich vom
                        Anwender aus in der selben Richtung befinden. Werden zwie Spieler getroffen, erhalte zwei
                        Treffer-Marker anstatt einen.\\
      \hline
      \textbf{Fluftext} & - \\
      \hline
    \end{tabular}
  \end{center}
\end{table}

\begin{table}[H]
  \begin{center}
    \label{tab:table1}
    \begin{tabular}{|l|p{8cm}|}
      \hline
      \textbf{Name} & Goldener Colt \\
      \hline
      \textbf{Art} & Fernkampf \\
      \hline
      \textbf{Munitionsverbrauch} & 1 \\
      \hline
      \textbf{Effekt} & Am Ende des Spiels + 500\$ \\
      \hline
      \textbf{Fluftext} & - \\
      \hline
    \end{tabular}
  \end{center}
\end{table}
