\section{Begriffe}

\begin{itemize}
    \item \textbf{Wagon} Als Wagon wird ein einzelnes Teil des Zuges bezeichnet. Der einfach heit halber wird die Lock auch als
            Wagon bezeichnet
    \item \textbf{Feld} Wagon haben zwei Felder. Ein Feld 'Auf dem Dach' und ein Feld 'Im inneren'
            Objekte die sich auf dem gleichen Wagon, aber einer anderen Ebene befinden, befinden sich NICHT
            auf dem gleichen Feld. Jedes Feld kann drei (bzw zwei bei Lock und letzem Wagon) angrenzende Felder haben.
            Diese sind die Felder in den nachbar Wagons auf der gleichen Eben und das Feld im selben Wagon auf der
            anderen Ebene.
    \item \textbf{Fallen lassen} Sofern nicht andert Beschrieben, wird das ensprechende Plätchen auf das Feld gelegt auf dem
            sich der betroffene Bandiet gerade befindet. Gibt es zu dem Plätchen eine dazugehörige Karte
            (z.b. bei Waffen), muss diese wieder in den ensprechenden Vorrat gelegt werden.
\end{itemize}